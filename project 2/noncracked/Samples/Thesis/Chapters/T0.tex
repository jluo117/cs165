% -----------------------------------------------------------------------------
% -*-TeX-*- -*-Hard-*- Smart Wrapping
% -----------------------------------------------------------------------------
%%% Thesis Introduction -------------------------------------------------------

\nonumchapter{Introduction}

The question
\begin{quote}
  {\em Does every bounded linear operator on a Banach space have a
  non--trivial closed invariant subspace? }
\end{quote}
is known as the {\em invariant subspace problem}.

\medskip

The examples due to Enflo~\cite{Enf87} and Read~\cite{Rea85} show that the
answer to the invariant subspace problem is in general negative. However,
there are no known examples of operators without invariant subspaces acting
on a reflexive Banach space and in particular, on a Hilbert space.
Furthermore, there seems to be no evidence what should be an expected answer
for the operators acting on a Hilbert space, and the experts in the field
have different opinions on it.

It is therefore not surprising that there are relatively few special cases
for which the existence of invariant subspaces have been established. One of
the most prominent results is the one on the existence of hyperinvariant
subspaces for compact operators due to Lomonosov~\cite{Lom73,RR73}. Another
class of operators that is well understood in terms of invariant subspaces
are normal, and in particular, self--adjoint operators for which there is the
powerful spectral theorem. However, it is not known whether a compact
perturbation of a self--adjoint operator has a non--trivial invariant
subspace.

This work focuses on the existence of invariant subspaces for essentially
self--adjoint operators and culminates in an affirmative answer in the case
where the underlying Hilbert space is assumed to be {\em real}.

\medskip

When dealing with the existence of invariant subspaces it is a common
practice~\cite{AAB95,dB93,Lom91} to study the space of certain continuous
functions associated with the algebra generated by an operator rather than
the operator itself. We follow this approach and establish a connection
between the existence of invariant subspaces for an operator algebra and
density of certain associated spaces of continuous functions called {\em
Lomonosov spaces}. The construction of these functions is based on the idea
of the {\em partition of unity} subordinate to an open cover, which is a
standard tool in approximation theory~\cite{Gam90} and differential
geometry~\cite{Dev68,Spi65}. In~\cite{Dev68} the partition of unity is also
used to prove the Arzela--Ascoli Theorem. It should be observed that similar
argument was employed by V.I.~Lomonosov in the proof of his celebrated
result~\cite{Lom73}.

On a Hilbert space differentiability of the norm yields a numerical criterion
for the construction of Lomonosov functions with certain properties. This
results in an extension of the Burnside Theorem and implies the solution of
what we define as the ``essentially--transitive algebra problem''.

An application of the extended Burnside Theorem to the algebra generated by
an essentially self--adjoint operator yields the existence of vector states
on the space of polynomials restricted to the essential spectrum of such an
operator. The invariant subspace problem for compact perturbations of
self--adjoint operators is translated into an extreme problem and the
solution (in the case where the underlying Hilbert space is real) is obtained
upon differentiating certain real--valued functions at their extreme.

Although the above--described techniques do not immediately extend to the
complex Hilbert spaces, it is very likely that further analysis of the space
of vector states will reduce the complex case to the real one and thus
provide the affirmative answer to one of the most difficult questions in the
theory of invariant subspaces~\cite{Lom92}.

%%% ---------------------------------------------------------------------------
